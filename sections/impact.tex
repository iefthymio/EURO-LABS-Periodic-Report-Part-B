%%%%%%%%%%%%%%%%%%%%%%%%%%%%%%%%%%%%%%%%%
%%% Explanation of progress towards delivering 
%%% scientific impact
%%%%%%%%%%%%%%%%%%%%%%%%%%%%%%%%%%%%%%%%%

\subsection{Impact}
\label{sec:impact}

%\todo{Include in this section whether the information on section 2.1 of the DoA  (how your project will contribute to the expected impacts) is still relevant or needs to be updated. Include further details in the latter case.}

In P2, EURO-LABS made substantial contributions through the provision of TA support to the project's leading Research Infrastructures (RIs). This support was pivotal for frontline experiments, research and development (R\&D), training young researchers, and advancing Open Science initiatives.

For \textbf{WP2}, the executed TA projects focused on high-priority topics in nuclear fundamental physics and its applications, in alignment with the recommendations set out in the NUPECC Long Range Plan. The outcomes of these efforts will be showcased through publications in scientific journals and conference proceedings. The service improvement activities undertaken in P2 have a lasting impact on the quality of services provided by several RIs. Notably, the biomedical FLASH project, which tested ultra-high dose rates with pencil beam scanning, stands out as a key example.

The facilities involved in \textbf{WP3} play a crucial role, serving as unique infrastructures for accelerator technology R\&D. These contribute to advancing our understanding of fundamental physics and are also positioned to drive innovations with wide-reaching implications beyond high-energy physics. The significance of these efforts clearly emerges from the experiments conducted and supported by TA funding during P2.

Next generation SRF cavities relies on the development of new superconducting thin films and multilayers to operate at higher temperature \SI{4.2}{K} while preserving the performances obtained on Nb at \SI{2}{K}. The CEA-IRFU/Synergium through his two platforms MACHAFILM and CRYOMECH serves as major test bed in the R\&D contributions to international accelerator project such as PIP II, ESS, ICONES with world recognized expertise in accelerators components design, development and construction. The research trusts encompass thin films, additive manufacturing and innovative closed-loop cooling schemes developments for next generation SRF cavities and magnets in order to increase their energy efficiency and reduce the dependence to critical resources (Helium). Within the EURO-LABS project and the Synergium, the MACHAFILM laboratory provides access to state of the art SRF cavity surface treatments facilities, characterization techniques and deposition methods routinely used such chemical and thermal treatments, tunneling spectroscopy and Atomic layer deposition. 


%\todo{Elaborate on the WP3 highlights including Applications - INCT and possibly CLEAR}

The majority of the executed TA projects of \textbf{WP4} are of vital importance for the successful completion of the HL-LHC project, the focal point of the European Strategy for Particle Physics (ESPP) in 2013 and its update in 2020. The remaining ones are synchronized with the stipulations of the ECFA Detector R\&D Roadmap, serving either the medium term goals, like the Higgs factory, or the more distant endeavor towards the highest energy hadron collider (FCC-hh). WP4 backs up the newly set up DRD collaborations, by providing them with access to top-level infrastructure needed for their detector R\&D. The more precise guideline to future facilities, e.g. which Higgs factory to build in the next decade, is expected from the currently ongoing strategy process (ESPP), which is to be adopted by the CERN Council in early 2026.

The overall project is on track delivering the services needed for detector R\&D. The resulting scientific impact is two-fold: scientific results in instrumentation are being published in scientific journals and conference proceedings during the duration of the project. The more glorious part with the physics results of the HL-LHC experiments and their follow-ups at future colliders will, however, start emerging during the next decade and beyond, when the detectors now being tested at WP4 RIs will be put to their proper usage and deliver precious physics data.

The activities in \textbf{WP5} are having a significant impact on the dissemination of EURO-LABS activities and scientific results (Task 5.1). The project’s website (https://web.infn.it/EURO-LABS/) serves as a vital communication tool, keeping the broader community involved in the project informed. To raise awareness of TA opportunities at various facilities, videos have been produced and are available on the website. EURO-LABS activities on social media, namely LinkedIn and Instagram, have started.  

Task 5.2 is focused on promoting FAIR research data management practices within the scientific community. New services have been introduced to help researchers discover and access datasets, including a catalog, authentication, and data access platforms. These efforts are increasing the visibility of experimental datasets, facilitate access and reuse, and foster collaboration among research groups. The Data Management Plan developed for the project and reviewed in P2 not only outlines practices within EURO-LABS but also encourages the adoption of FAIR practices at the participating RIs.

Task 5.3 is dedicated to developing new techniques that can be applied across multiple RIs, with the goal of enhancing scientific output at large. One example is the Generic Optimization Framework and Frontend (GeOFF), which is being implemented at various accelerator facilities throughout Europe. This tool aims to automate accelerator control systems, improving the availability of machines like the Fragment Separator (FRS) for physics experiments. The new system reduces the time required for FRS setup from 2–3 days to just a few hours. GeOFF will also be adapted to the specific needs of the Laser-Plasma Accelerator (LPA) at the LPA-UHI100 facility in WP3, streamlining the optimization phase for laser-driven sources and helping to maintain optimal operating conditions for the accelerator. These developments are anticipated to significantly enhance the scientific output of laser-driven acceleration experiments.

Finally, Task 5.4 focuses on training young researchers, thereby strengthening the global scientific impact of European research teams and ensuring the future of nuclear and particle physics research. 
%The first training event took place in September 2023, with strong interest from young researchers who responded quickly to the announcement made in April. 
Four training events have been organized in P2 (for 2023 and 2024). These training activities are vital for fostering the knowledge and enthusiasm of the next generation, ensuring the continued advancement of the field.

