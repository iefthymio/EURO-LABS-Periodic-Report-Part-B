
\section{List of Critical Risks}

\subsection{Foreseen risks}

\projnote{In P1 report the table appears with the three additional columns on the right. However in the templeate these three columns should be reported as part of the table in Section 4.3 below}

\def\arraystretch{1.25}
\arrayrulecolor{mygray}
% {\fontsize{9}{11}\selectfont
\setlength{\arrayrulewidth}{0.5pt} % Adjust border width (optional)
\arrayrulecolor{mygray} % Set border color
\begin{xltabular}{\hsize}{|c|p{0.2\hsize}|p{0.1\hsize}|p{0.3\hsize}|X|X|X|} \hline

    \rowcolor{mylightergray}\multicolumn{7}{|l|}{\textbf{Foreseen risks}} \\ \hline
    \rowcolor{mylightergray}
    \textbf{Risk No} & 
    \textbf{Description} & 
    \textbf{Work package No(s)} & 
    \textbf{Proposed mitigation} &
    \textbf{Did you apply risk mitigation measures?} &
    \textbf{Did your risk materialise?} &
    \textbf{State of the play Comments} 
    \\ \hline
    \endfirsthead
        
    % \multicolumn{7}{|l|}% {-- continued from previous page} \\ 
    \hline 
    \rowcolor{mylightergray}
    \textbf{Risk No} & 
    \textbf{Description} & 
    \textbf{Work package No(s)} & 
    \textbf{Proposed mitigation} &
    \textbf{Did you apply risk mitigation measures?} &
    \textbf{Did your risk materialise?} &
    \textbf{State of the play Comments} 
    \\ \hline
    \endhead
    
    \hline \multicolumn{7}{|r|}{{Continues on next page}} \\ \hline
    \endfoot

    1 &	
    Covid-19 or other pandemic-related risks: travel constraints, confinement, recurrence of pandemic (low-medium likelihood, medium severity) & 
    1,2,3,4,5	&
    Measures to ensure safe working conditions under confinement situation, have been realized and already adopted in the past. Possibility to follow-up and intervene in experiments and their preparation from off-site locations thus limiting the travel requirements. More local involvement to mitigate the reduced number of participants on-site &
    No & No No & \\ \hline

    2 &
    Breakdown of specific components of accelerators (very low likelihood, impact could be medium/high)	&
    2	&
    The planned activities (and the related allocated budget) could be shifted to other facilities in the consortium. &
    No & No  & No\\ \hline
    
    3	&
    Closure of ECT*: i) low, ii) high	&
    2	&
    Maintain relations with local and international funding agencies, nuclear physics community and other stakeholders &
    No & No  & No\\ \hline
    
    4	&
    Reduced availability of RIs due to longer shutdowns or unforeseen technical stops (Medium/Medium)	&
    2,3,4	&
    Reschedule TAs for later times if possible, otherwise rearrange tests to accommodate more in parallel. If not possible to resolve within the RI, shift access units to other RIs starting from within the same Task, then same WP, and eventually in other WPs. &
    No & No  & No\\ \hline
    \pagebreak
    
    5 &
    Allocated user fails to realize the planned experiment/test and TAs (Medium/Medium)	&
    2,3,4	&
    Establish confirmation milestone for each user for the TA allocation. Reschedule for a later time slot in the facility. Propose the allocated slot to other users. &
    No & No  & No\\ \hline
    
    6	& 
    Failure to attract the foreseen number of users to the TA facilities (Low/High)	&
    2,3,4	&
    Regular monitoring of TA allocation within WP/Task. Effort for better publicity of the access opportunities offered by the RIs, promote success stories among target user community for the RI. Use dynamic allocation of access units (i.e. shift of EC funds to other RIs) within the WP/Task. &
    No & No  & No\\ \hline
    
    7	&
    Failure to complete the planned service improvements (Low/Medium)	&
    2,3,4 &
    Regular reporting of the progress and accomplishment of milestones. Investigate of possible alternative service improvements having similar positive impact for the users. &
    No & No  & No\\ \hline
    
    8	&
    Change of management team or WP coordinators or FC during the project (Medium/Medium) &
    1,2,3,4,5	&
    Anticipate potential staff changed in the project management and WP/FC coordination and select suitable replacements within the participating scientists of the RI and the consortium for higher management positions as early as possible &
    No & No  & No\\ \hline
    
    9 &
    Prolonged accelerator breakdown caused by major malfunctions (Medium/Low)	&
    2	&
    Because of accelerators variety in EURO-LABS, major accidents in one machine will not expected to jeopardize the programs that can be carried out in other accelerator facilities. &
    No & No  & No\\ \hline
    \pagebreak
    
    10	& 
    Due to variations in the needs of the participating facilities, the scope of the
    remote operations toolkit may be too large to be reasonably included in the database within the given time frame (Medium/Low)	&
    2	&
    A manageable subset of the most critical items that would allow all participating facilities to carry out remote operation at an acceptable level would then be selected and implemented in the database. &
    No & No  & No\\ \hline
    
    11	&
    Delay in the recruitment of the post-doc researchers (Low/Low)	&
    2	&
    The post-doc positions are required in three different institutions, one 12-month position for each. So, a delay in some of the institutions will not affect the recruitment of the other institutions. The request for the recruitment will be submitted at the beginning of the project so the planned activity can be completed within the three-year project. &
    No & No  & No\\ \hline
\end{xltabular}
% }

\newpage
\subsection{Unforeseen risks}

\def\arraystretch{1.25}
\arrayrulecolor{mygray}
% {\fontsize{9}{11}\selectfont
\setlength{\arrayrulewidth}{0.5pt} % Adjust border width (optional)
\arrayrulecolor{mygray} % Set border color
\begin{xltabular}{\hsize}{|c|p{0.2\hsize}|p{0.1\hsize}|p{0.3\hsize}|X|X|X|} \hline

    \rowcolor{mylightergray}\multicolumn{7}{|l|}{\textbf{Unforeseen risks}} \\ \hline
    \rowcolor{mylightergray}
    \textbf{Risk No} & 
    \textbf{Description} & 
    \textbf{Work package No(s)} & 
    \textbf{Proposed mitigation} &
    \textbf{Did you apply risk mitigation measures?} &
    \textbf{Did your risk materialise?} &
    \textbf{State of the play Comments} 
    \\ \hline
    \endfirsthead
        
    % \multicolumn{7}{|l|}% {-- continued from previous page} \\ 
    \hline 
    \rowcolor{mylightergray}
    \textbf{Risk No} & 
    \textbf{Description} & 
    \textbf{Work package No(s)} & 
    \textbf{Proposed mitigation} &
    \textbf{Did you apply risk mitigation measures?} &
    \textbf{Did your risk materialise?} &
    \textbf{State of the play Comments} 
    \\ \hline
    \endhead
    
    \hline \multicolumn{7}{|r|}{{Continues on next page}} \\ \hline
    \endfoot

    1 &	2 & 3 & 4 & 5 & 6 & 7 \\ \hline
  
\end{xltabular}
% }

\newpage

\subsection{State of play}

\def\arraystretch{1.25}
\arrayrulecolor{mygray}
% {\fontsize{9}{11}\selectfont
\setlength{\arrayrulewidth}{0.5pt} % Adjust border width (optional)
\arrayrulecolor{mygray} % Set border color
\begin{xltabular}{\hsize}{|c|p{0.2\hsize}|p{0.1\hsize}|p{0.3\hsize}|X|} \hline

    \rowcolor{mylightergray}\multicolumn{5}{|l|}{\textbf{State of play}} \\ \hline
    \rowcolor{mylightergray}
    \textbf{Risk No} & 
    \textbf{Reporting period} & 
    \textbf{Did you apply risk mitigation measures?} &
    \textbf{Did your risk materialise?} &
    \textbf{Comments} 
    \\ \hline
    \endfirsthead
        
    % \multicolumn{7}{|l|}% {-- continued from previous page} \\ 
    \hline 
    \rowcolor{mylightergray}
    \textbf{Risk No} & 
    \textbf{Reporting period} & 
    \textbf{Did you apply risk mitigation measures?} &
    \textbf{Did your risk materialise?} &
    \textbf{Comments} 
    \\ \hline
    \endhead
    
    \hline \multicolumn{5}{|r|}{{Continues on next page}} \\ \hline
    \endfoot

    1 &	2 & 3 & 4 & 5 \\ \hline
  
\end{xltabular}