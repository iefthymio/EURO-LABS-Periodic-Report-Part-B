%
\clearpage
\section{Open Science}
As part of the EURO-LABS project, a range of targeted initiatives have been implemented to promote and enhance open science practices within the associated research communities.

To begin with, an open-access repository has been established on the Zenodo platform to host the project’s outputs: https://zenodo.org/communities/euro-labs/records. A variety of digital objects—such as deliverables, presentations, posters, and source code—have already been successfully deposited, in accordance with the data management guidelines outlined under Work Package 5.2 (WP5.2).

Additionally, WP5.2 has delivered a service to the community that provides an Authorization and Authentication Infrastructure (AAI) based on the IAM solution. This system has been successfully integrated into several services, both within the EURO-LABS framework—such as Theo4NP, which offers computational infrastructure, and the OpenNP catalog—and beyond, including authentication for logging and graphing services at GANIL and collaborative platforms at GSI/FAIR.

Finally, in November 2024, a basic training school was organized with the support of WP5.4, aimed at educating the next generation of stakeholders—including PhD students, postdoctoral researchers, research staff, and data managers—on open science and data management topics. Notably, the concept for this school emerged during the course of the project, driven by the collaborative interactions among project members.