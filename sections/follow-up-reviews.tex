%%%%%%%%%%%%%%%%%%%%%%%%%%%%%%%%%%%%%%%%%
%%% Follow-up of Recommendations and 
%%% Comments from Previous Review(s) 
%%%%%%%%%%%%%%%%%%%%%%%%%%%%%%%%%%%%%%%%%

\clearpage
\section[Follow-up of Recommendations and Comments from Previous Review(s)]{\texorpdfstring{Follow-up of Recommendations and Comments\\from Previous Review(s)}{Follow-up of Recommendations and Comments from Previous Review(s)}}
\label{sec:follow-up-reviews}

%%%%%%%%%%%%%%%%%%%%%%%%%%%%%%%%%%%%%%%%%
%%% Section content, please change!
%%%%%%%%%%%%%%%%%%%%%%%%%%%%%%%%%%%%%%%%%

\todo{Include in this section the list of recommendations and comments from previous reviews and give information on how they have been followed up.}

%%%%%%%%%%%%%%%%%%%%%%%%%%%%%%%%%%%%%%%%%
{\it Resource Allocation.}
The project office has taken all relevant actions regarding the appropriate use of resources.

{\it Dissemination and Communication.}
To improve dissemination and communication—which initially relied on relatively conservative methods—the recommended steps have been implemented. Dedicated LinkedIn, Instagram, and YouTube channels have been launched. In addition, a media professional has been engaged to enhance these efforts and develop a comprehensive communication plan.

Videos highlighting the facilities available for transnational access are now accessible via the Transnational Access section on the EURO-LABS website. Each facility employs its own approach to attract users and promote scientific outreach: several laboratories offer programs for school students, allowing them to participate in specific activities within various divisions, in addition to providing general overviews of the facilities and engagement in science-related but not strictly scientific activities.

A biannual newsletter showcases project updates, highlights key activities, and provides information on upcoming events.

{\it Transnational Access Funding.}
Minor corrections concerning the reallocation and procedural handling of funds for transnational access have been discussed within both the Steering Committee and the Governing Board. Final decisions regarding reallocation are expected to be made during the upcoming Governing Board meeting in autumn this year.

{\it Industrial Users.}
Many aspects related to industrial users fall outside the project's scope. However, facilities such as GANIL and JYFL, for example, are successfully attracting industrial users, although these users often pursue commercial or classified objectives.

 \textbf{Further inputs need to be provided. What can we add for WP3 and WP4?}

{\it IPR and Outreach}
The scientific and technological activities supported by the project do not include Intellectual Property Rights (IPR), as they are intended to be open-access. This is a key reason why defense and industrial exploitation activities are not included.

All presentations and outputs acknowledge EURO-LABS support, and the project has gained recognition also within communities not directly involved. Invited talks have been delivered beyond Europe, emphasizing the unique nature of the Horizon Europe framework and the global distinctiveness of EURO-LABS. Talks are often presented at major international conferences.

{\it Future of the Project.}
Given the highly positive experience and performance of EURO-LABS, the Steering Committee had already agreed that the three core communities would submit a joint proposal once the draft for the Horizon Europe Work Programme 2025 became available—almost a year ago.

The initial draft versions of the call (250625 and 221024) closely aligned with the EURO-LABS model and also proposed the inclusion of the hadron physics community. However, the proposed budget was reduced by approximately 31\% compared to that of EURO-LABS.

Recently, the Programme Committee, for reasons not fully clarified, revised the call's focus exclusively toward hadron physics (aligned with STRONG2020). This change was officially communicated to the relevant national ministries in early April. Under these new conditions, the globally unique EURO-LABS project will not be continued.

The Steering Committee has yet to discuss the appropriate course of action in detail; this will be a key agenda item at the next Governing Board meeting.


