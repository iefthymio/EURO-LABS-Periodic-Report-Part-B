%%%%%%%%%%%%%%%%%%%%%%%%%%%%%%%%%%%%%%%%%
%%% Explanation of the Work carried out
%%% by the Beneficiaries and Overview of 
%%% the Progress
% This section consists of several sub-sections that will be published through CORDIS and possibly other communication channels. Your text should be easy to read, that is, written in an understandable and % accessible way for a broader public. Its purpose should aim to promote the dissemination and support the % exploitation of EU funded results. Altogether, your text should not exceed 7480 characters. You should refer only to publicly available information and must not include any confidential or personal data (e.g. names and addresses). The project summary (for publication) must be drafted as a "stand-alone" text. No references should be made to other parts of the report. You may also wish to provide diagrams or photographs illustrating and promoting the work of your project (only as images).
%%%%%%%%%%%%%%%%%%%%%%%%%%%%%%%%%%%%%%%%%

\section{Project Summary (for publication)}


\subsection*{Context and overall objectives}

Nuclear and High Energy Physics employ diverse approaches across a wide range of energy scales, into the understanding of the composition of the Universe and elucidating how the behaviour of its fundamental elements could explain what we observe today.  This pursuit necessitates information from exploring what happens at the smallest scale, from the atomic nuclei down to the quark level. And it relies on the continual advancement of cutting-edge technologies in accelerators, detectors and test facilities enabling precise measurements to be realised, leading to the development of theoretical models for key questions.

EUROpean Laboratories for Accelerator Based Sciences (EURO-LABS) brings together for the first time the high and low-energy communities. It serves as the inaugural effort to enhance communication and synergies between these communities and build a super community of European researchers and technical staff to address various current challenges and build the foundation for world leading endeavours, thereby injecting vitality into the field of science and enrich its links to society. Its primary aim is to facilitate streamlined access to both existing and emerging state-of-the-art Research Infrastructures (RIs) across Europe, and allow a diverse community of users to conduct high-impact research addressing challenges in both physics and technology. EURO-LABS promotes knowledge sharing across scientific domains and contributes to the advancement of scientific and technical frontier knowledge in Europe. To boost the competitiveness of European RIs, a coordinating board was established to plan and implement training programs. In addition to lectures, each training school will provide hands-on-training in accelerator-based science and will be hosted at specific RIs, capitalizing on their unique strengths and capabilities. They are categorized into basic and advanced schools for researchers, and also a school dedicated to technical staff.  Another important endeavour is to enhance data accessibility within the project, adhering to the FAIR principles. Open science presents a new opportunity for the nuclear physics community to improve the management and use of the extensive datasets produced throughout Europe, a practice already well established in the high-energy physics community.  This initiative will stimulate new collaborations and integrate the nuclear physics community into the European Open Science Cloud (EOSC) framework.

\projnote{Note: Maybe we could add a summary here of the work done and reported in P1?}

\subsection*{Work performed from M13 (1 September 2023) to M30 (28 February 2025) and main results achieved}

The Project Office (PO), consisting of members from INFN Bologna, including one full-time experienced support staff for daily operations and planning, along with INFN Frascati and CERN EU offices (lending their expertise on specialized tasks) was formed. The PO together with the Steering Committee (SC) and the Management Team (MT) ensure the smooth running of the technical part and the related administration of the project. A website (https://web.infn.it/EURO-LABS/) was created and provides all information about the project serving also as gateway to the wide range of facilities in EURO-LABS, current events, available employment etc. The Kick Off Meeting for the project was held in Bologna, a month after the official project started (https://agenda.infn.it/event/32088/). It was attended by approx. 85 participants in person, an excellent opportunity for communication and getting to know each other scientifically. Every task and work package leader gave an overview of the corresponding actions planned during the project and their timelines. The meeting concluded with the first meeting of the Governing Board.  
The progress in the various work packages of the project is going on schedule as planned. All the due milestone and deliverables were completed on time. Regarding transnational access to the participating RIs, the various User Selection Panels (USP) have been established, mandated to oversee the distribution of the access funds in all the 47 facilities and RIs of the project. Calls for beam time at the various facilities were advertised by the individual facilities, and through the EURO-LABS website, which also provides detailed information on the procedure for availing the modalities of Trans-national Access (TA) or Virtual Access (VA) support. In the selection process efforts have been made to encourage and promote younger and new users and those with limited abilities. Overall, the TA is proceeding on plan, with a large number of projects completed and access beam hours delivered.  As expected, at this start-up period we observed variations in the delivery profile of access between facilities. Some surpassed the rough initial year’s projection (~25\% access units), with overbooking. There were isolated cases where the access delivered so far was below the projection, primarily due to delays in ongoing known upgrades or other technical reasons. These variations are expected to smooth out nevertheless and the SC is monitoring the situation and will take corrective measures if required.  The number of female users was 26\% with a lower average age than the male users. The service improvements proposed to elevate the quality of offered services at various RIs are progressing as planned. These improvements are expected to be completed within the next two years and be used within the duration of the project. Most of the recruitment positions has been successfully accomplished. Virtual access to the new Theo4Exp infrastructure is ahead of schedule.


\subsection*{Results beyond the state of the art}

The project has actively engaged the diverse facilities in conveying project-related activities to the general public, young school students and, at times, local and national government bodies. This included open days, organized laboratory visits, demonstrations at science days and fairs, presented in an accessible and engaging manner. The funded access to the various facilities in this project is leading to improved skills of young researchers. Numerous projects were selected and executed, providing the users with valuable resources in their strive for science and technology, producing already high-impact scientific results. Open science is a new opportunity for the nuclear physics community to improve the management and use of the extensive datasets produced throughout Europe, a practice already established in the high-energy community. The development of innovative methods such as the utilisation of machine learning algorithms for accelerator beam control and optimization, and for the management of ion sources in laser-driven accelerators, is actively underway.  The focus is on the development of open tools and platforms, e.g., an accessible beam diagnostic database and an optimizer toolkit.  These toolkits can be adapted for use at various accelerators.  In one case, beam tuning time has been reduced from 8 hours to 10 minutes! Such achievements contribute to maximizing the data collected during experiments and have the potential to improve the environmental carbon footprint of the facilities. Finally, the planned hands-on schools dedicated to students, young post-docs and engineers are a way to amplify the competitiveness of the RIs and their technical capabilities to forge a coherent, stable, and predictable future of European facilities.

\subsection*{Policy reletance of your project (if applicable)}

Not applicable.