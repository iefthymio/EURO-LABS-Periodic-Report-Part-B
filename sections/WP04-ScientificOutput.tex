%
% 
\subparagraph{Work Package 4 - Access to RIs for Detectors} \mbox{}

WP4 is the enabler of Detector R\&D, the fundament of current and future scientific endavour. Most of its scientific output is thus hidden, displaced to the developed detectors, their implementation in experiments and the fundamental physics discoveries those experiments shall unravel. Needless to list what shall be discovered – if we knew that, there would be no need for conducting experiments.

WP4 is of vital importance for the successful completion of the HL-LHC project, the focal point of the European Strategy for Particle Physics (ESPP) in 2013 and its update in 2020. The upgrades have nearly finished their R\&D and the construction of the detectors is in full swing. There are, however, late surprises that still require verification of the solutions in the WP4 RI’s. The experiment upgrades for the LHC are amply documented in the respective Technical Design Reports (TDR) and in the Memoranda of Understanding (MoU) for the specific detectors, available on the CERN Document Server (CDS).

The shift of focus in WP4 is synchronized with the stipulations of the ECFA Detector R\&D Roadmap, serving either the medium term goals like the Higgs factory or the more distant endeavour towards the highest energy hadron collider (FCC-hh). WP4 backs up the newly set up DRD collaborations by providing them with access to top-level infrastructure needed for their detector R\&D. The more precise guideline to future facilities, e.g. which Higgs factory to build in the next decade is expected from the currently ongoing strategy process (ESPP), to be adopted by the CERN Council in early 2026.

The spectrum of DRD collaboration research activities is best illustrated in their proposals and their structuring into Work Packages and Working Groups. They follow closely the programme, set up by  the DRD Themes (DRDT) specified in the Detector R\&D Roadmap.

WP4 is delivering the services needed for detector R\&D. The resulting scientific output in terms of publications is two-fold: scientific results in instrumentation are being published in scientific journals and conference proceedings during the duration of the project. The more glorious part with the physics results of the HL-LHC experiments and their follow-ups at future colliders will, however, start emerging during the next decade and beyond, when the detectors now being tested at WP4 RIs will be put to their proper usage and deliver precious physics data.
